\setcounter{section}{1}
\section{Hatvány, Gyök, Logarithmus}

\subsection{Hatvány}

\dfn{Hatvány} {
	\begin{enumerate}[label=\bfseries\tiny\protect\circled{\small\arabic*}]
		\item Ha $n\in\left(\NN^+ \textbackslash\{1\}\right)$:\\
			$a^n$ egy $n$ tényezős szorzat, amelynek minden tényezője $a$
		\item Ha $n=1$:
			$$a^1=1$$
		\item Ha $n=0$:
			\begin{center}
				$a^0=1$, $\left(a\neq0\right)$
			\end{center}
		\item Ha $n\in\NN^+$:
			$$a^{-n} = \frac{1}{a^n} \left(a\neq 0\right)$$
	\end{enumerate}
}
\nt{Később $n\in\ZZ$ és $n\in\QQ^*$-ra is kiterjesztjük}

\begin{tcolorbox}
	\begin{center}
		\bf\huge{Azonosságok}
	\end{center}
	\begin{minipage}{0.49\textwidth}
		\begin{tcolorbox}[width=\textwidth,center]
			$$\left(a^n\right)^m=a^{n\cdot m}$$
		\end{tcolorbox}
		\begin{tcolorbox}[width=\textwidth,center]
			$$a^n\cdot a^m = a^{n+m}$$
		\end{tcolorbox}
	\end{minipage}
	\hfill
	\begin{minipage}{0.49\textwidth}
		\begin{tcolorbox}[width=\textwidth,center]
			$$\frac{a^n}{a^m}=a^{n-m}$$
		\end{tcolorbox}
		\begin{tcolorbox}[width=\textwidth,center]
			$$\left(a\cdot b\right)^n = a^n\cdot b^n$$
		\end{tcolorbox}
	\end{minipage}
	\begin{tcolorbox}[width=0.5\textwidth,center]$$\left(\frac{a}{b}\right)^n = \frac{a^n}{b^n}$$\end{tcolorbox}
\end{tcolorbox}

\thm{Permanecia Elv}{A definició bővítés során az eddigi azonosságok általánosságára kell törekedni}

\newpage
\subsection{Gyök}

\dfn{Négyzetgyök} {
	Az $x$ nem-negatív szám négyzetgyöke az a nem-negatív szám, amelynek négyzete $x$.
	$$\left(\sqrt{x}\right)^2 = x (x\geq 0)$$
	$$\sqrt[2]{x}\geq 0$$
}

\begin{tcolorbox}
	\begin{center}
		\bf\huge{Azonosságok}
	\end{center}
	\begin{minipage}{0.49\textwidth}
		\begin{tcolorbox}[width=\textwidth,center]
			$$\sqrt{a}\cdot\sqrt{b}=\sqrt{a\cdot b}$$
			\kikot{$a\geq 0$ és $b\geq 0$}
		\end{tcolorbox}
	\end{minipage}
	\hfill
	\begin{minipage}{0.49\textwidth}
		\begin{tcolorbox}[width=\textwidth,center]
			$$\frac{\sqrt{a}}{\sqrt{b}} = \sqrt{\frac{a}{b}}$$
			\kikot{$a\geq 0$ és $b > 0$}
		\end{tcolorbox}
	\end{minipage}
	\begin{tcolorbox}[width=0.5\textwidth,center]$$\sqrt{a^b} = \left(\sqrt{a}\right)^b$$ \kikot{$a\geq 0$}\end{tcolorbox}
\end{tcolorbox}

\dfn{n-edik gyök} {
	\begin{enumerate}[label=\bfseries\tiny\protect\circled{\small\arabic*}]
		\item Ha $n=2k\left(k\in\NN^+\right)$, akkor:\\
			Az $x$ nem-negatív szám $2k$-adik gyöke az a nem-negatív szám, mely $2k$-ra emelve $x$.
		\item Ha $n=2k+1\left(k\in\NN^+\right)$, akkor:\\
			Az $x$ valós szám $2k+1$-edik gyöke az a valós szám, mely $2k+1$-edik hatványkitevője $x$.
	\end{enumerate}
}

\ex{}{
\begin{align}
	\sqrt[3]{64}&=4\\
	\sqrt[3]{-64}&=-4
\end{align}
}

\begin{tcolorbox}
	\begin{center}
		\bf\huge{Azonosságok}
	\end{center}
	\begin{minipage}{0.49\textwidth}
		\begin{tcolorbox}[width=\textwidth,center]
			$$\sqrt[n]{a}\cdot\sqrt[n]{b}=\sqrt[n]{a\cdot b}$$
		\end{tcolorbox}
		\begin{tcolorbox}[width=\textwidth,center]
			$$\sqrt[n]{a^b} = \left(\sqrt[n]{a}\right)^b$$
		\end{tcolorbox}
	\end{minipage}
	\hfill
	\begin{minipage}{0.49\textwidth}
		\begin{tcolorbox}[width=\textwidth,center]
			$$\frac{\sqrt[n]{a}}{\sqrt[n]{b}} = \sqrt[n]{\frac{a}{b}} (b\neq 0)$$
		\end{tcolorbox}
		\begin{tcolorbox}[width=\textwidth,center]
			$$\sqrt[n]{\sqrt[m]{a}} = \sqrt[n\cdot m]{a}$$
		\end{tcolorbox}
	\end{minipage}
\end{tcolorbox}


\newpage
\subsection{Logarithmus}


